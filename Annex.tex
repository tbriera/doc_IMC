
\begin{landscape}
    \section{Annex}
\subsection{Technical and economic assumptions for electricity generation}

%%%%%%%%%%%%%%%%%%%%%%%%%%%%%%%%%%%%%%%%%%%%%%%%%%%%%%%%%%%%%%%%
%%%%%%%%%%%%%%%%%%%%%%%%%%%%%%%%%%%%%%%%%%%%%%%%%%%%%%%%%%%%%%%%
% CINV_MW_ref : CAPEX

%%% Data with no/sensible source
%% Old POLES data: OCT, OGC
\begingroup\fontsize{8}{10}\selectfont

\begin{ThreePartTable}
\begin{TableNotes}[para]
\item \underline{\textit{Sources:}} 
\item For renewables, Renewable Power Generation Costs in 2019 (IRENA, 2020), when data for 2014 is available. When it is not, the 2019 data is used to go backwards on the learning curve, assuming the relationship between the installed capacity and the investment costs hold for the years 2014-2019 . For other technologies (except oil and biomass), Power generation assumptions in the Stated Policies and SDS Scenarios in the World Energy Outlook 2021 - Assumptions for additional technologies and regions in the Stated Policies and Net Zero scenarios (IEA, 2021). For biomass, Cost development of low carbon energy technologies (JRC, 2018). For oil, Lazard's Levelized Cost of Energy Analysis version 8.0 (Lazard, 2014).
\end{TableNotes}
\begin{longtable}[t]{lrrrrrrrrrrrrrrrrrrrrr}
\caption{CAPEX, in thousand 2010\$ per MW}\\
\toprule
 & PFC & PSS & ICG & CGS & SUB & USC & UCS & GGT & GGS & GGC & OCT & OGC & HYD & NUC & CSP & WND & WNO & CPV & RPV & BIGCC & BIGCCS\\
\midrule
\cellcolor{gray!6}{USA} & \cellcolor{gray!6}{1948.0} & \cellcolor{gray!6}{4731} & \cellcolor{gray!6}{2412} & \cellcolor{gray!6}{5195} & \cellcolor{gray!6}{1670.0} & \cellcolor{gray!6}{2134.0} & \cellcolor{gray!6}{4916} & \cellcolor{gray!6}{463.8} & \cellcolor{gray!6}{2783} & \cellcolor{gray!6}{927.6} & \cellcolor{gray!6}{650} & \cellcolor{gray!6}{991.7} & \cellcolor{gray!6}{2505} & \cellcolor{gray!6}{4638} & \cellcolor{gray!6}{6030} & \cellcolor{gray!6}{1766} & \cellcolor{gray!6}{5470} & \cellcolor{gray!6}{2676} & \cellcolor{gray!6}{3315} & \cellcolor{gray!6}{3457} & \cellcolor{gray!6}{6240}\\
CAN & 1948.0 & 4731 & 2412 & 5195 & 1670.0 & 2134.0 & 4916 & 463.8 & 2783 & 927.6 & 650 & 991.7 & 2505 & 4638 & 6030 & 2266 & 5470 & 2676 & 3315 & 3457 & 6240\\
\cellcolor{gray!6}{EUR} & \cellcolor{gray!6}{1855.0} & \cellcolor{gray!6}{4638} & \cellcolor{gray!6}{2319} & \cellcolor{gray!6}{5102} & \cellcolor{gray!6}{1577.0} & \cellcolor{gray!6}{2041.0} & \cellcolor{gray!6}{4824} & \cellcolor{gray!6}{463.8} & \cellcolor{gray!6}{2783} & \cellcolor{gray!6}{927.6} & \cellcolor{gray!6}{650} & \cellcolor{gray!6}{991.7} & \cellcolor{gray!6}{2458} & \cellcolor{gray!6}{5566} & \cellcolor{gray!6}{5241} & \cellcolor{gray!6}{1865} & \cellcolor{gray!6}{4879} & \cellcolor{gray!6}{2198} & \cellcolor{gray!6}{2672} & \cellcolor{gray!6}{3457} & \cellcolor{gray!6}{6240}\\
OECD & 2226.0 & 5009 & 2690 & 5473 & 1948.0 & 2412.0 & 5195 & 463.8 & 2876 & 1020.0 & 650 & 991.7 & 2226 & 3896 & 6030 & 2708 & 5264 & 2789 & 2640 & 3457 & 6240\\
\cellcolor{gray!6}{FSU} & \cellcolor{gray!6}{1855.0} & \cellcolor{gray!6}{4638} & \cellcolor{gray!6}{2319} & \cellcolor{gray!6}{5102} & \cellcolor{gray!6}{1577.0} & \cellcolor{gray!6}{2041.0} & \cellcolor{gray!6}{4824} & \cellcolor{gray!6}{417.4} & \cellcolor{gray!6}{2597} & \cellcolor{gray!6}{742.1} & \cellcolor{gray!6}{650} & \cellcolor{gray!6}{991.7} & \cellcolor{gray!6}{2458} & \cellcolor{gray!6}{3525} & \cellcolor{gray!6}{5241} & \cellcolor{gray!6}{1865} & \cellcolor{gray!6}{6163} & \cellcolor{gray!6}{2198} & \cellcolor{gray!6}{2672} & \cellcolor{gray!6}{3457} & \cellcolor{gray!6}{6240}\\
\addlinespace
CHN & 649.3 & 3432 & 1020 & 3803 & 556.6 & 742.1 & 3525 & 324.7 & 2375 & 519.5 & 650 & 991.7 & 1484 & 2412 & 4545 & 1252 & 3852 & 1635 & 1541 & 3457 & 6240\\
\cellcolor{gray!6}{IND} & \cellcolor{gray!6}{1113.0} & \cellcolor{gray!6}{3896} & \cellcolor{gray!6}{1484} & \cellcolor{gray!6}{4267} & \cellcolor{gray!6}{927.6} & \cellcolor{gray!6}{1299.0} & \cellcolor{gray!6}{4082} & \cellcolor{gray!6}{371.0} & \cellcolor{gray!6}{2505} & \cellcolor{gray!6}{649.3} & \cellcolor{gray!6}{650} & \cellcolor{gray!6}{991.7} & \cellcolor{gray!6}{1855} & \cellcolor{gray!6}{2597} & \cellcolor{gray!6}{5287} & \cellcolor{gray!6}{1316} & \cellcolor{gray!6}{4031} & \cellcolor{gray!6}{1769} & \cellcolor{gray!6}{1541} & \cellcolor{gray!6}{3457} & \cellcolor{gray!6}{6240}\\
BRA & 1484.0 & 4267 & 1855 & 4638 & 1206.0 & 1670.0 & 4453 & 371.0 & 2505 & 649.3 & 650 & 991.7 & 1948 & 3710 & 4963 & 2061 & 5932 & 3623 & 3438 & 3457 & 6240\\
\cellcolor{gray!6}{MDE} & \cellcolor{gray!6}{1484.0} & \cellcolor{gray!6}{4267} & \cellcolor{gray!6}{1855} & \cellcolor{gray!6}{4638} & \cellcolor{gray!6}{1206.0} & \cellcolor{gray!6}{1484.0} & \cellcolor{gray!6}{4267} & \cellcolor{gray!6}{417.4} & \cellcolor{gray!6}{2597} & \cellcolor{gray!6}{742.1} & \cellcolor{gray!6}{650} & \cellcolor{gray!6}{991.7} & \cellcolor{gray!6}{1994} & \cellcolor{gray!6}{3247} & \cellcolor{gray!6}{4870} & \cellcolor{gray!6}{1770} & \cellcolor{gray!6}{5881} & \cellcolor{gray!6}{2157} & \cellcolor{gray!6}{3537} & \cellcolor{gray!6}{3457} & \cellcolor{gray!6}{6240}\\
AFR & 1484.0 & 4267 & 2041 & 4824 & 1206.0 & 1762.0 & 4545 & 371.0 & 2505 & 649.3 & 650 & 991.7 & 1948 & 3710 & 4684 & 1930 & 5701 & 2543 & 3496 & 3457 & 6240\\
\addlinespace
\cellcolor{gray!6}{RAS} & \cellcolor{gray!6}{649.3} & \cellcolor{gray!6}{3432} & \cellcolor{gray!6}{1020} & \cellcolor{gray!6}{3803} & \cellcolor{gray!6}{556.6} & \cellcolor{gray!6}{742.1} & \cellcolor{gray!6}{3525} & \cellcolor{gray!6}{324.7} & \cellcolor{gray!6}{2375} & \cellcolor{gray!6}{519.5} & \cellcolor{gray!6}{650} & \cellcolor{gray!6}{991.7} & \cellcolor{gray!6}{1484} & \cellcolor{gray!6}{2412} & \cellcolor{gray!6}{4545} & \cellcolor{gray!6}{1252} & \cellcolor{gray!6}{3852} & \cellcolor{gray!6}{1635} & \cellcolor{gray!6}{1541} & \cellcolor{gray!6}{3457} & \cellcolor{gray!6}{6240}\\
RAL & 1484.0 & 4267 & 1855 & 4638 & 1206.0 & 1670.0 & 4453 & 371.0 & 2505 & 649.3 & 650 & 991.7 & 1948 & 3710 & 4963 & 2061 & 5932 & 2617 & 2638 & 3457 & 6240\\
\bottomrule
\insertTableNotes
\end{longtable}
\end{ThreePartTable}
\endgroup{}


% O&M_fix : fixed O&M
%% Old POLES data : OCT, OGC, HYD
\begingroup\fontsize{8}{10}\selectfont

\begin{ThreePartTable}
\begin{TableNotes}[para]
\item \underline{\textit{Sources:}} 
\item NREL (2010) Cost and Performance Assumptions for Modeling Electricity Generation Technologies, IEA (2020) World Energy Outlook 2020
\end{TableNotes}
\begin{longtable}[t]{lrrrrrrrrrrrrrrrrrrrrr}
\caption{Fixed O\&M, in thousand 2010\$ per MW}\\
\toprule
 & PFC & PSS & ICG & CGS & SUB & USC & UCS & GGT & GGS & GGC & OCT & OGC & HYD & NUC & CSP & WND & WNO & CPV & RPV & BIGCC & BIGCCS\\
\midrule
\cellcolor{gray!6}{USA} & \cellcolor{gray!6}{55.87} & \cellcolor{gray!6}{146.1} & \cellcolor{gray!6}{77.36} & \cellcolor{gray!6}{180.5} & \cellcolor{gray!6}{38.68} & \cellcolor{gray!6}{60.17} & \cellcolor{gray!6}{146.1} & \cellcolor{gray!6}{17.19} & \cellcolor{gray!6}{180.50} & \cellcolor{gray!6}{21.49} & \cellcolor{gray!6}{39.26} & \cellcolor{gray!6}{69.11} & \cellcolor{gray!6}{56.73} & \cellcolor{gray!6}{150.4} & \cellcolor{gray!6}{223.5} & \cellcolor{gray!6}{32.66} & \cellcolor{gray!6}{111.70} & \cellcolor{gray!6}{15.47} & \cellcolor{gray!6}{44.70} & \cellcolor{gray!6}{0} & \cellcolor{gray!6}{0}\\
CAN & 55.87 & 146.1 & 77.36 & 180.5 & 38.68 & 60.17 & 146.1 & 17.19 & 180.50 & 21.49 & 39.26 & 69.11 & 56.73 & 150.4 & 223.5 & 32.66 & 111.70 & 15.47 & 44.70 & 0 & 0\\
\cellcolor{gray!6}{EUR} & \cellcolor{gray!6}{51.57} & \cellcolor{gray!6}{141.8} & \cellcolor{gray!6}{77.36} & \cellcolor{gray!6}{176.2} & \cellcolor{gray!6}{38.68} & \cellcolor{gray!6}{51.57} & \cellcolor{gray!6}{141.8} & \cellcolor{gray!6}{17.19} & \cellcolor{gray!6}{176.20} & \cellcolor{gray!6}{21.49} & \cellcolor{gray!6}{39.26} & \cellcolor{gray!6}{69.11} & \cellcolor{gray!6}{56.73} & \cellcolor{gray!6}{137.5} & \cellcolor{gray!6}{197.7} & \cellcolor{gray!6}{34.38} & \cellcolor{gray!6}{64.47} & \cellcolor{gray!6}{10.31} & \cellcolor{gray!6}{15.47} & \cellcolor{gray!6}{0} & \cellcolor{gray!6}{0}\\
JAN & 60.17 & 150.4 & 85.96 & 189.1 & 47.28 & 60.17 & 150.4 & 17.19 & 189.10 & 25.79 & 39.26 & 69.11 & 56.73 & 193.4 & 223.5 & 48.14 & 68.76 & 27.51 & 25.79 & 0 & 0\\
\cellcolor{gray!6}{CEI} & \cellcolor{gray!6}{60.17} & \cellcolor{gray!6}{154.7} & \cellcolor{gray!6}{77.36} & \cellcolor{gray!6}{163.3} & \cellcolor{gray!6}{42.98} & \cellcolor{gray!6}{60.17} & \cellcolor{gray!6}{154.7} & \cellcolor{gray!6}{21.49} & \cellcolor{gray!6}{163.30} & \cellcolor{gray!6}{25.79} & \cellcolor{gray!6}{39.26} & \cellcolor{gray!6}{69.11} & \cellcolor{gray!6}{56.73} & \cellcolor{gray!6}{137.5} & \cellcolor{gray!6}{197.7} & \cellcolor{gray!6}{34.38} & \cellcolor{gray!6}{103.10} & \cellcolor{gray!6}{27.51} & \cellcolor{gray!6}{36.10} & \cellcolor{gray!6}{0} & \cellcolor{gray!6}{0}\\
\addlinespace
CHN & 25.79 & 120.3 & 42.98 & 141.8 & 17.19 & 25.79 & 120.3 & 17.19 & 141.80 & 17.19 & 39.26 & 69.11 & 56.73 & 103.1 & 171.9 & 25.79 & 64.47 & 10.31 & 12.03 & 0 & 0\\
\cellcolor{gray!6}{IND} & \cellcolor{gray!6}{42.98} & \cellcolor{gray!6}{133.2} & \cellcolor{gray!6}{60.17} & \cellcolor{gray!6}{159.0} & \cellcolor{gray!6}{30.08} & \cellcolor{gray!6}{42.98} & \cellcolor{gray!6}{133.2} & \cellcolor{gray!6}{17.19} & \cellcolor{gray!6}{159.00} & \cellcolor{gray!6}{21.49} & \cellcolor{gray!6}{39.26} & \cellcolor{gray!6}{69.11} & \cellcolor{gray!6}{56.73} & \cellcolor{gray!6}{120.3} & \cellcolor{gray!6}{197.7} & \cellcolor{gray!6}{22.35} & \cellcolor{gray!6}{55.87} & \cellcolor{gray!6}{10.31} & \cellcolor{gray!6}{10.31} & \cellcolor{gray!6}{0} & \cellcolor{gray!6}{0}\\
BRA & 55.87 & 120.3 & 77.36 & 141.8 & 38.68 & 55.87 & 120.3 & 17.19 & 141.80 & 21.49 & 39.26 & 69.11 & 56.73 & 146.1 & 180.5 & 32.66 & 98.85 & 15.47 & 15.47 & 0 & 0\\
\cellcolor{gray!6}{MDE} & \cellcolor{gray!6}{55.87} & \cellcolor{gray!6}{120.3} & \cellcolor{gray!6}{77.36} & \cellcolor{gray!6}{141.8} & \cellcolor{gray!6}{38.68} & \cellcolor{gray!6}{55.87} & \cellcolor{gray!6}{120.3} & \cellcolor{gray!6}{21.49} & \cellcolor{gray!6}{141.80} & \cellcolor{gray!6}{25.79} & \cellcolor{gray!6}{39.26} & \cellcolor{gray!6}{69.11} & \cellcolor{gray!6}{56.73} & \cellcolor{gray!6}{137.5} & \cellcolor{gray!6}{180.5} & \cellcolor{gray!6}{39.54} & \cellcolor{gray!6}{98.85} & \cellcolor{gray!6}{12.03} & \cellcolor{gray!6}{20.63} & \cellcolor{gray!6}{0} & \cellcolor{gray!6}{0}\\
AFR & 51.57 & 150.4 & 77.36 & 180.5 & 38.68 & 51.57 & 150.4 & 17.19 & 159.00 & 21.49 & 39.26 & 69.11 & 56.73 & 146.1 & 171.9 & 41.26 & 94.55 & 20.63 & 27.51 & 0 & 0\\
\addlinespace
\cellcolor{gray!6}{RAS} & \cellcolor{gray!6}{25.79} & \cellcolor{gray!6}{120.3} & \cellcolor{gray!6}{42.98} & \cellcolor{gray!6}{141.8} & \cellcolor{gray!6}{17.19} & \cellcolor{gray!6}{25.79} & \cellcolor{gray!6}{120.3} & \cellcolor{gray!6}{17.19} & \cellcolor{gray!6}{55.87} & \cellcolor{gray!6}{17.19} & \cellcolor{gray!6}{39.26} & \cellcolor{gray!6}{69.11} & \cellcolor{gray!6}{56.73} & \cellcolor{gray!6}{103.1} & \cellcolor{gray!6}{171.9} & \cellcolor{gray!6}{25.79} & \cellcolor{gray!6}{64.47} & \cellcolor{gray!6}{10.31} & \cellcolor{gray!6}{12.03} & \cellcolor{gray!6}{0} & \cellcolor{gray!6}{0}\\
RAL & 55.87 & 120.3 & 77.36 & 141.8 & 38.68 & 55.87 & 120.3 & 17.19 & 141.80 & 21.49 & 39.26 & 69.11 & 56.73 & 146.1 & 180.5 & 32.66 & 98.85 & 15.47 & 15.47 & 0 & 0\\
\bottomrule
\insertTableNotes
\end{longtable}
\end{ThreePartTable}
\endgroup{}



% O&M_var : variable O&M
%% Hyp for OCT and OGC == GGT and GGC
\begingroup\fontsize{8}{10}\selectfont

\begin{ThreePartTable}
\begin{TableNotes}[para]
\item \underline{\textit{Sources:}} 
\item NREL (2010) Cost and Performance Assumptions for Modeling Electricity Generation Technologies
\end{TableNotes}
\begin{longtable}[t]{lrrrrrrrrrrrrrrrrrrrrr}
\caption{Variable O\&M, in 2010\$ per MWh}\\
\toprule
 & PFC & PSS & ICG & CGS & SUB & USC & UCS & GGT & GGS & GGC & OCT & OGC & HYD & NUC & CSP & WND & WNO & CPV & RPV & BIGCC & BIGCCS\\
\midrule
\cellcolor{gray!6}{USA} & \cellcolor{gray!6}{0.01} & \cellcolor{gray!6}{0.01} & \cellcolor{gray!6}{0} & \cellcolor{gray!6}{0} & \cellcolor{gray!6}{0.01} & \cellcolor{gray!6}{0.01} & \cellcolor{gray!6}{0.01} & \cellcolor{gray!6}{0} & \cellcolor{gray!6}{0} & \cellcolor{gray!6}{0} & \cellcolor{gray!6}{0} & \cellcolor{gray!6}{0} & \cellcolor{gray!6}{0} & \cellcolor{gray!6}{0} & \cellcolor{gray!6}{0} & \cellcolor{gray!6}{0} & \cellcolor{gray!6}{0} & \cellcolor{gray!6}{0} & \cellcolor{gray!6}{0} & \cellcolor{gray!6}{0} & \cellcolor{gray!6}{0}\\
CAN & 0.01 & 0.01 & 0 & 0 & 0.01 & 0.01 & 0.01 & 0 & 0 & 0 & 0 & 0 & 0 & 0 & 0 & 0 & 0 & 0 & 0 & 0 & 0\\
\cellcolor{gray!6}{EUR} & \cellcolor{gray!6}{0.01} & \cellcolor{gray!6}{0.01} & \cellcolor{gray!6}{0} & \cellcolor{gray!6}{0} & \cellcolor{gray!6}{0.01} & \cellcolor{gray!6}{0.01} & \cellcolor{gray!6}{0.01} & \cellcolor{gray!6}{0} & \cellcolor{gray!6}{0} & \cellcolor{gray!6}{0} & \cellcolor{gray!6}{0} & \cellcolor{gray!6}{0} & \cellcolor{gray!6}{0} & \cellcolor{gray!6}{0} & \cellcolor{gray!6}{0} & \cellcolor{gray!6}{0} & \cellcolor{gray!6}{0} & \cellcolor{gray!6}{0} & \cellcolor{gray!6}{0} & \cellcolor{gray!6}{0} & \cellcolor{gray!6}{0}\\
JAN & 0.01 & 0.01 & 0 & 0 & 0.01 & 0.01 & 0.01 & 0 & 0 & 0 & 0 & 0 & 0 & 0 & 0 & 0 & 0 & 0 & 0 & 0 & 0\\
\cellcolor{gray!6}{CEI} & \cellcolor{gray!6}{0.01} & \cellcolor{gray!6}{0.01} & \cellcolor{gray!6}{0} & \cellcolor{gray!6}{0} & \cellcolor{gray!6}{0.01} & \cellcolor{gray!6}{0.01} & \cellcolor{gray!6}{0.01} & \cellcolor{gray!6}{0} & \cellcolor{gray!6}{0} & \cellcolor{gray!6}{0} & \cellcolor{gray!6}{0} & \cellcolor{gray!6}{0} & \cellcolor{gray!6}{0} & \cellcolor{gray!6}{0} & \cellcolor{gray!6}{0} & \cellcolor{gray!6}{0} & \cellcolor{gray!6}{0} & \cellcolor{gray!6}{0} & \cellcolor{gray!6}{0} & \cellcolor{gray!6}{0} & \cellcolor{gray!6}{0}\\
\addlinespace
CHN & 0.01 & 0.01 & 0 & 0 & 0.01 & 0.01 & 0.01 & 0 & 0 & 0 & 0 & 0 & 0 & 0 & 0 & 0 & 0 & 0 & 0 & 0 & 0\\
\cellcolor{gray!6}{IND} & \cellcolor{gray!6}{0.01} & \cellcolor{gray!6}{0.01} & \cellcolor{gray!6}{0} & \cellcolor{gray!6}{0} & \cellcolor{gray!6}{0.01} & \cellcolor{gray!6}{0.01} & \cellcolor{gray!6}{0.01} & \cellcolor{gray!6}{0} & \cellcolor{gray!6}{0} & \cellcolor{gray!6}{0} & \cellcolor{gray!6}{0} & \cellcolor{gray!6}{0} & \cellcolor{gray!6}{0} & \cellcolor{gray!6}{0} & \cellcolor{gray!6}{0} & \cellcolor{gray!6}{0} & \cellcolor{gray!6}{0} & \cellcolor{gray!6}{0} & \cellcolor{gray!6}{0} & \cellcolor{gray!6}{0} & \cellcolor{gray!6}{0}\\
BRA & 0.01 & 0.01 & 0 & 0 & 0.01 & 0.01 & 0.01 & 0 & 0 & 0 & 0 & 0 & 0 & 0 & 0 & 0 & 0 & 0 & 0 & 0 & 0\\
\cellcolor{gray!6}{MDE} & \cellcolor{gray!6}{0.01} & \cellcolor{gray!6}{0.01} & \cellcolor{gray!6}{0} & \cellcolor{gray!6}{0} & \cellcolor{gray!6}{0.01} & \cellcolor{gray!6}{0.01} & \cellcolor{gray!6}{0.01} & \cellcolor{gray!6}{0} & \cellcolor{gray!6}{0} & \cellcolor{gray!6}{0} & \cellcolor{gray!6}{0} & \cellcolor{gray!6}{0} & \cellcolor{gray!6}{0} & \cellcolor{gray!6}{0} & \cellcolor{gray!6}{0} & \cellcolor{gray!6}{0} & \cellcolor{gray!6}{0} & \cellcolor{gray!6}{0} & \cellcolor{gray!6}{0} & \cellcolor{gray!6}{0} & \cellcolor{gray!6}{0}\\
AFR & 0.01 & 0.01 & 0 & 0 & 0.01 & 0.01 & 0.01 & 0 & 0 & 0 & 0 & 0 & 0 & 0 & 0 & 0 & 0 & 0 & 0 & 0 & 0\\
\addlinespace
\cellcolor{gray!6}{RAS} & \cellcolor{gray!6}{0.01} & \cellcolor{gray!6}{0.01} & \cellcolor{gray!6}{0} & \cellcolor{gray!6}{0} & \cellcolor{gray!6}{0.01} & \cellcolor{gray!6}{0.01} & \cellcolor{gray!6}{0.01} & \cellcolor{gray!6}{0} & \cellcolor{gray!6}{0} & \cellcolor{gray!6}{0} & \cellcolor{gray!6}{0} & \cellcolor{gray!6}{0} & \cellcolor{gray!6}{0} & \cellcolor{gray!6}{0} & \cellcolor{gray!6}{0} & \cellcolor{gray!6}{0} & \cellcolor{gray!6}{0} & \cellcolor{gray!6}{0} & \cellcolor{gray!6}{0} & \cellcolor{gray!6}{0} & \cellcolor{gray!6}{0}\\
RAL & 0.01 & 0.01 & 0 & 0 & 0.01 & 0.01 & 0.01 & 0 & 0 & 0 & 0 & 0 & 0 & 0 & 0 & 0 & 0 & 0 & 0 & 0 & 0\\
\bottomrule
\insertTableNotes
\end{longtable}
\end{ThreePartTable}
\endgroup{}


% Rho : energy efficiency
%%  old POLES data : OCT, OGC
\begingroup\fontsize{8}{10}\selectfont

\begin{ThreePartTable}
\begin{TableNotes}[para]
\item \underline{\textit{Sources:}} 
\item Power generation assumptions in the Stated Policies and SDS Scenarios in the World Energy Outlook 2020 - Assumptions for additional technologies and regions in the Stated Policies and Net Zero scenarios (IEA, 2020). For oil : Lazard's Levelized Cost of Energy Analysis version 8.0 (Lazard, 2014).
\end{TableNotes}
\begin{longtable}[t]{lrrrrrrrrrrrrrrrrrrrrr}
\caption{Energy efficiency (rho), in \%}\\
\toprule
 & PFC & PSS & ICG & CGS & SUB & USC & UCS & GGT & GGS & GGC & OCT & OGC & HYD & NUC & CSP & WND & WNO & CPV & RPV & BIGCC & BIGCCS\\
\midrule
\cellcolor{gray!6}{USA} & \cellcolor{gray!6}{0.43} & \cellcolor{gray!6}{0.36} & \cellcolor{gray!6}{0.44} & \cellcolor{gray!6}{0.36} & \cellcolor{gray!6}{0.39} & \cellcolor{gray!6}{0.45} & \cellcolor{gray!6}{0.38} & \cellcolor{gray!6}{0.40} & \cellcolor{gray!6}{0.51} & \cellcolor{gray!6}{0.59} & \cellcolor{gray!6}{0.34} & \cellcolor{gray!6}{0.45} & \cellcolor{gray!6}{1} & \cellcolor{gray!6}{0.36} & \cellcolor{gray!6}{0} & \cellcolor{gray!6}{0} & \cellcolor{gray!6}{0} & \cellcolor{gray!6}{0} & \cellcolor{gray!6}{0} & \cellcolor{gray!6}{0.4} & \cellcolor{gray!6}{0.3}\\
CAN & 0.43 & 0.36 & 0.44 & 0.36 & 0.39 & 0.45 & 0.38 & 0.40 & 0.51 & 0.59 & 0.34 & 0.45 & 1 & 0.36 & 0 & 0 & 0 & 0 & 0 & 0.4 & 0.3\\
\cellcolor{gray!6}{EUR} & \cellcolor{gray!6}{0.43} & \cellcolor{gray!6}{0.36} & \cellcolor{gray!6}{0.44} & \cellcolor{gray!6}{0.36} & \cellcolor{gray!6}{0.39} & \cellcolor{gray!6}{0.45} & \cellcolor{gray!6}{0.38} & \cellcolor{gray!6}{0.40} & \cellcolor{gray!6}{0.51} & \cellcolor{gray!6}{0.59} & \cellcolor{gray!6}{0.34} & \cellcolor{gray!6}{0.45} & \cellcolor{gray!6}{1} & \cellcolor{gray!6}{0.36} & \cellcolor{gray!6}{0} & \cellcolor{gray!6}{0} & \cellcolor{gray!6}{0} & \cellcolor{gray!6}{0} & \cellcolor{gray!6}{0} & \cellcolor{gray!6}{0.4} & \cellcolor{gray!6}{0.3}\\
OECD & 0.43 & 0.36 & 0.44 & 0.36 & 0.39 & 0.45 & 0.38 & 0.40 & 0.51 & 0.59 & 0.34 & 0.45 & 1 & 0.36 & 0 & 0 & 0 & 0 & 0 & 0.4 & 0.3\\
\cellcolor{gray!6}{FSU} & \cellcolor{gray!6}{0.43} & \cellcolor{gray!6}{0.36} & \cellcolor{gray!6}{0.44} & \cellcolor{gray!6}{0.36} & \cellcolor{gray!6}{0.39} & \cellcolor{gray!6}{0.45} & \cellcolor{gray!6}{0.38} & \cellcolor{gray!6}{0.38} & \cellcolor{gray!6}{0.49} & \cellcolor{gray!6}{0.57} & \cellcolor{gray!6}{0.34} & \cellcolor{gray!6}{0.45} & \cellcolor{gray!6}{1} & \cellcolor{gray!6}{0.36} & \cellcolor{gray!6}{0} & \cellcolor{gray!6}{0} & \cellcolor{gray!6}{0} & \cellcolor{gray!6}{0} & \cellcolor{gray!6}{0} & \cellcolor{gray!6}{0.4} & \cellcolor{gray!6}{0.3}\\
\addlinespace
CHN & 0.41 & 0.35 & 0.43 & 0.36 & 0.37 & 0.44 & 0.37 & 0.38 & 0.49 & 0.57 & 0.34 & 0.45 & 1 & 0.36 & 0 & 0 & 0 & 0 & 0 & 0.4 & 0.3\\
\cellcolor{gray!6}{IND} & \cellcolor{gray!6}{0.40} & \cellcolor{gray!6}{0.31} & \cellcolor{gray!6}{0.41} & \cellcolor{gray!6}{0.36} & \cellcolor{gray!6}{0.36} & \cellcolor{gray!6}{0.40} & \cellcolor{gray!6}{0.33} & \cellcolor{gray!6}{0.38} & \cellcolor{gray!6}{0.48} & \cellcolor{gray!6}{0.56} & \cellcolor{gray!6}{0.34} & \cellcolor{gray!6}{0.45} & \cellcolor{gray!6}{1} & \cellcolor{gray!6}{0.36} & \cellcolor{gray!6}{0} & \cellcolor{gray!6}{0} & \cellcolor{gray!6}{0} & \cellcolor{gray!6}{0} & \cellcolor{gray!6}{0} & \cellcolor{gray!6}{0.4} & \cellcolor{gray!6}{0.3}\\
BRA & 0.43 & 0.35 & 0.44 & 0.36 & 0.39 & 0.45 & 0.37 & 0.38 & 0.49 & 0.58 & 0.34 & 0.45 & 1 & 0.36 & 0 & 0 & 0 & 0 & 0 & 0.4 & 0.3\\
\cellcolor{gray!6}{MDE} & \cellcolor{gray!6}{0.41} & \cellcolor{gray!6}{0.35} & \cellcolor{gray!6}{0.42} & \cellcolor{gray!6}{0.36} & \cellcolor{gray!6}{0.37} & \cellcolor{gray!6}{0.43} & \cellcolor{gray!6}{0.37} & \cellcolor{gray!6}{0.38} & \cellcolor{gray!6}{0.49} & \cellcolor{gray!6}{0.57} & \cellcolor{gray!6}{0.34} & \cellcolor{gray!6}{0.45} & \cellcolor{gray!6}{1} & \cellcolor{gray!6}{0.36} & \cellcolor{gray!6}{0} & \cellcolor{gray!6}{0} & \cellcolor{gray!6}{0} & \cellcolor{gray!6}{0} & \cellcolor{gray!6}{0} & \cellcolor{gray!6}{0.4} & \cellcolor{gray!6}{0.3}\\
AFR & 0.39 & 0.32 & 0.40 & 0.36 & 0.35 & 0.42 & 0.34 & 0.38 & 0.50 & 0.58 & 0.34 & 0.45 & 1 & 0.36 & 0 & 0 & 0 & 0 & 0 & 0.4 & 0.3\\
\addlinespace
\cellcolor{gray!6}{RAS} & \cellcolor{gray!6}{0.41} & \cellcolor{gray!6}{0.35} & \cellcolor{gray!6}{0.43} & \cellcolor{gray!6}{0.36} & \cellcolor{gray!6}{0.37} & \cellcolor{gray!6}{0.44} & \cellcolor{gray!6}{0.37} & \cellcolor{gray!6}{0.38} & \cellcolor{gray!6}{0.49} & \cellcolor{gray!6}{0.57} & \cellcolor{gray!6}{0.34} & \cellcolor{gray!6}{0.45} & \cellcolor{gray!6}{1} & \cellcolor{gray!6}{0.36} & \cellcolor{gray!6}{0} & \cellcolor{gray!6}{0} & \cellcolor{gray!6}{0} & \cellcolor{gray!6}{0} & \cellcolor{gray!6}{0} & \cellcolor{gray!6}{0.4} & \cellcolor{gray!6}{0.3}\\
RAL & 0.43 & 0.35 & 0.44 & 0.36 & 0.39 & 0.45 & 0.37 & 0.38 & 0.49 & 0.58 & 0.34 & 0.45 & 1 & 0.36 & 0 & 0 & 0 & 0 & 0 & 0.4 & 0.3\\
\bottomrule
\insertTableNotes
\end{longtable}
\end{ThreePartTable}
\endgroup{}



% avail_load : available/load factors
%% hydro : POLES data, author's calculation
\begingroup\fontsize{8}{10}\selectfont

\begin{ThreePartTable}
\begin{TableNotes}[para]
\item \underline{\textit{Sources:}} 
\item Power generation assumptions in the Stated Policies and SDS Scenarios in the World Energy Outlook 2021 - Assumptions for additional technologies and regions in the Stated Policies and Net Zero scenarios (IEA, 2020). Australia values used for the OECD Pacific region from Projected Cost of Generating Electricity (IEA/NEA, 2020) when available (WND,WNO,CPV,CSP). For hydro, POLES model data, author's calculation.
\end{TableNotes}
\begin{longtable}[t]{lrrrrrrrrrrrrrrrrrrrrr}
\caption{Availability factor (for dispatchable plants)/ Load factor (for variable renewable plants), in \% }\\
\toprule
 & PFC & PSS & ICG & CGS & SUB & USC & UCS & GGT & GGS & GGC & OCT & OGC & HYD & NUC & CSP & WND & WNO & CPV & RPV & BIGCC & BIGCCS\\
\midrule
\cellcolor{gray!6}{USA} & \cellcolor{gray!6}{0.83} & \cellcolor{gray!6}{0.83} & \cellcolor{gray!6}{0.82} & \cellcolor{gray!6}{0.82} & \cellcolor{gray!6}{0.83} & \cellcolor{gray!6}{0.83} & \cellcolor{gray!6}{0.83} & \cellcolor{gray!6}{0.91} & \cellcolor{gray!6}{0.84} & \cellcolor{gray!6}{0.84} & \cellcolor{gray!6}{0.91} & \cellcolor{gray!6}{0.84} & \cellcolor{gray!6}{0.37} & \cellcolor{gray!6}{1} & \cellcolor{gray!6}{0.28} & \cellcolor{gray!6}{0.42} & \cellcolor{gray!6}{0.41} & \cellcolor{gray!6}{0.21} & \cellcolor{gray!6}{0.16} & \cellcolor{gray!6}{0.83} & \cellcolor{gray!6}{0.83}\\
CAN & 0.83 & 0.83 & 0.82 & 0.82 & 0.83 & 0.83 & 0.83 & 0.91 & 0.84 & 0.84 & 0.91 & 0.84 & 0.56 & 1 & 0.30 & 0.42 & 0.41 & 0.13 & 0.11 & 0.83 & 0.83\\
\cellcolor{gray!6}{EUR} & \cellcolor{gray!6}{0.83} & \cellcolor{gray!6}{0.83} & \cellcolor{gray!6}{0.82} & \cellcolor{gray!6}{0.82} & \cellcolor{gray!6}{0.83} & \cellcolor{gray!6}{0.83} & \cellcolor{gray!6}{0.83} & \cellcolor{gray!6}{0.91} & \cellcolor{gray!6}{0.84} & \cellcolor{gray!6}{0.84} & \cellcolor{gray!6}{0.91} & \cellcolor{gray!6}{0.84} & \cellcolor{gray!6}{0.39} & \cellcolor{gray!6}{1} & \cellcolor{gray!6}{0.30} & \cellcolor{gray!6}{0.28} & \cellcolor{gray!6}{0.49} & \cellcolor{gray!6}{0.13} & \cellcolor{gray!6}{0.11} & \cellcolor{gray!6}{0.83} & \cellcolor{gray!6}{0.83}\\
OECD & 0.83 & 0.83 & 0.82 & 0.82 & 0.83 & 0.83 & 0.83 & 0.91 & 0.84 & 0.84 & 0.91 & 0.84 & 0.34 & 1 & 0.38 & 0.34 & 0.45 & 0.20 & 0.14 & 0.83 & 0.83\\
\cellcolor{gray!6}{FSU} & \cellcolor{gray!6}{0.83} & \cellcolor{gray!6}{0.83} & \cellcolor{gray!6}{0.82} & \cellcolor{gray!6}{0.82} & \cellcolor{gray!6}{0.83} & \cellcolor{gray!6}{0.83} & \cellcolor{gray!6}{0.83} & \cellcolor{gray!6}{0.91} & \cellcolor{gray!6}{0.84} & \cellcolor{gray!6}{0.84} & \cellcolor{gray!6}{0.91} & \cellcolor{gray!6}{0.84} & \cellcolor{gray!6}{0.39} & \cellcolor{gray!6}{1} & \cellcolor{gray!6}{0.30} & \cellcolor{gray!6}{0.25} & \cellcolor{gray!6}{0.37} & \cellcolor{gray!6}{0.12} & \cellcolor{gray!6}{0.09} & \cellcolor{gray!6}{0.83} & \cellcolor{gray!6}{0.83}\\
\addlinespace
CHN & 0.83 & 0.83 & 0.82 & 0.82 & 0.83 & 0.83 & 0.83 & 0.91 & 0.84 & 0.84 & 0.91 & 0.84 & 0.39 & 1 & 0.28 & 0.25 & 0.32 & 0.17 & 0.13 & 0.83 & 0.83\\
\cellcolor{gray!6}{IND} & \cellcolor{gray!6}{0.83} & \cellcolor{gray!6}{0.83} & \cellcolor{gray!6}{0.82} & \cellcolor{gray!6}{0.82} & \cellcolor{gray!6}{0.83} & \cellcolor{gray!6}{0.83} & \cellcolor{gray!6}{0.83} & \cellcolor{gray!6}{0.91} & \cellcolor{gray!6}{0.84} & \cellcolor{gray!6}{0.84} & \cellcolor{gray!6}{0.91} & \cellcolor{gray!6}{0.84} & \cellcolor{gray!6}{0.37} & \cellcolor{gray!6}{1} & \cellcolor{gray!6}{0.26} & \cellcolor{gray!6}{0.26} & \cellcolor{gray!6}{0.29} & \cellcolor{gray!6}{0.20} & \cellcolor{gray!6}{0.15} & \cellcolor{gray!6}{0.83} & \cellcolor{gray!6}{0.83}\\
BRA & 0.83 & 0.83 & 0.82 & 0.82 & 0.83 & 0.83 & 0.83 & 0.91 & 0.84 & 0.84 & 0.91 & 0.84 & 0.48 & 1 & 0.28 & 0.44 & 0.46 & 0.20 & 0.16 & 0.83 & 0.83\\
\cellcolor{gray!6}{MDE} & \cellcolor{gray!6}{0.83} & \cellcolor{gray!6}{0.83} & \cellcolor{gray!6}{0.82} & \cellcolor{gray!6}{0.82} & \cellcolor{gray!6}{0.83} & \cellcolor{gray!6}{0.83} & \cellcolor{gray!6}{0.83} & \cellcolor{gray!6}{0.91} & \cellcolor{gray!6}{0.84} & \cellcolor{gray!6}{0.84} & \cellcolor{gray!6}{0.91} & \cellcolor{gray!6}{0.84} & \cellcolor{gray!6}{0.18} & \cellcolor{gray!6}{1} & \cellcolor{gray!6}{0.30} & \cellcolor{gray!6}{0.30} & \cellcolor{gray!6}{0.32} & \cellcolor{gray!6}{0.21} & \cellcolor{gray!6}{0.17} & \cellcolor{gray!6}{0.83} & \cellcolor{gray!6}{0.83}\\
AFR & 0.83 & 0.83 & 0.82 & 0.82 & 0.83 & 0.83 & 0.83 & 0.91 & 0.84 & 0.84 & 0.91 & 0.84 & 0.46 & 1 & 0.30 & 0.26 & 0.37 & 0.21 & 0.17 & 0.83 & 0.83\\
\addlinespace
\cellcolor{gray!6}{RAS} & \cellcolor{gray!6}{0.83} & \cellcolor{gray!6}{0.83} & \cellcolor{gray!6}{0.82} & \cellcolor{gray!6}{0.82} & \cellcolor{gray!6}{0.83} & \cellcolor{gray!6}{0.83} & \cellcolor{gray!6}{0.83} & \cellcolor{gray!6}{0.91} & \cellcolor{gray!6}{0.84} & \cellcolor{gray!6}{0.84} & \cellcolor{gray!6}{0.91} & \cellcolor{gray!6}{0.84} & \cellcolor{gray!6}{0.34} & \cellcolor{gray!6}{1} & \cellcolor{gray!6}{0.28} & \cellcolor{gray!6}{0.25} & \cellcolor{gray!6}{0.32} & \cellcolor{gray!6}{0.17} & \cellcolor{gray!6}{0.13} & \cellcolor{gray!6}{0.83} & \cellcolor{gray!6}{0.83}\\
RAL & 0.83 & 0.83 & 0.82 & 0.82 & 0.83 & 0.83 & 0.83 & 0.91 & 0.84 & 0.84 & 0.91 & 0.84 & 0.52 & 1 & 0.28 & 0.44 & 0.46 & 0.20 & 0.16 & 0.83 & 0.83\\
\bottomrule
\insertTableNotes
\end{longtable}
\end{ThreePartTable}
\endgroup{}


% Life_time : Lifetimes
%% Old POLES data : OCT, OGC
\begingroup\fontsize{8}{10}\selectfont

\begin{ThreePartTable}
\begin{TableNotes}[para]
\item \underline{\textit{Sources:}} 
\item IEA (2020) Projected Cost of Generating Electricity
\end{TableNotes}
\begin{longtable}[t]{lrrrrrrrrrrrrrrrrrrrrr}
\caption{Lifetimes, in year}\\
\toprule
 & PFC & PSS & ICG & CGS & SUB & USC & UCS & GGT & GGS & GGC & OCT & OGC & HYD & NUC & CSP & WND & WNO & CPV & RPV & BIGCC & BIGCCS\\
\midrule
\cellcolor{gray!6}{USA} & \cellcolor{gray!6}{40} & \cellcolor{gray!6}{40} & \cellcolor{gray!6}{40} & \cellcolor{gray!6}{40} & \cellcolor{gray!6}{40} & \cellcolor{gray!6}{40} & \cellcolor{gray!6}{40} & \cellcolor{gray!6}{30} & \cellcolor{gray!6}{30} & \cellcolor{gray!6}{30} & \cellcolor{gray!6}{25} & \cellcolor{gray!6}{25} & \cellcolor{gray!6}{80} & \cellcolor{gray!6}{60} & \cellcolor{gray!6}{25} & \cellcolor{gray!6}{25} & \cellcolor{gray!6}{25} & \cellcolor{gray!6}{25} & \cellcolor{gray!6}{25} & \cellcolor{gray!6}{15} & \cellcolor{gray!6}{15}\\
CAN & 40 & 40 & 40 & 40 & 40 & 40 & 40 & 30 & 30 & 30 & 25 & 25 & 80 & 60 & 25 & 25 & 25 & 25 & 25 & 15 & 15\\
\cellcolor{gray!6}{EUR} & \cellcolor{gray!6}{40} & \cellcolor{gray!6}{40} & \cellcolor{gray!6}{40} & \cellcolor{gray!6}{40} & \cellcolor{gray!6}{40} & \cellcolor{gray!6}{40} & \cellcolor{gray!6}{40} & \cellcolor{gray!6}{30} & \cellcolor{gray!6}{30} & \cellcolor{gray!6}{30} & \cellcolor{gray!6}{25} & \cellcolor{gray!6}{25} & \cellcolor{gray!6}{80} & \cellcolor{gray!6}{60} & \cellcolor{gray!6}{25} & \cellcolor{gray!6}{25} & \cellcolor{gray!6}{25} & \cellcolor{gray!6}{25} & \cellcolor{gray!6}{25} & \cellcolor{gray!6}{15} & \cellcolor{gray!6}{15}\\
JAN & 40 & 40 & 40 & 40 & 40 & 40 & 40 & 30 & 30 & 30 & 25 & 25 & 80 & 60 & 25 & 25 & 25 & 25 & 25 & 15 & 15\\
\cellcolor{gray!6}{CEI} & \cellcolor{gray!6}{40} & \cellcolor{gray!6}{40} & \cellcolor{gray!6}{40} & \cellcolor{gray!6}{40} & \cellcolor{gray!6}{40} & \cellcolor{gray!6}{40} & \cellcolor{gray!6}{40} & \cellcolor{gray!6}{30} & \cellcolor{gray!6}{30} & \cellcolor{gray!6}{30} & \cellcolor{gray!6}{25} & \cellcolor{gray!6}{25} & \cellcolor{gray!6}{80} & \cellcolor{gray!6}{60} & \cellcolor{gray!6}{25} & \cellcolor{gray!6}{25} & \cellcolor{gray!6}{25} & \cellcolor{gray!6}{25} & \cellcolor{gray!6}{25} & \cellcolor{gray!6}{15} & \cellcolor{gray!6}{15}\\
\addlinespace
CHN & 40 & 40 & 40 & 40 & 40 & 40 & 40 & 30 & 30 & 30 & 25 & 25 & 80 & 60 & 25 & 25 & 25 & 25 & 25 & 15 & 15\\
\cellcolor{gray!6}{IND} & \cellcolor{gray!6}{40} & \cellcolor{gray!6}{40} & \cellcolor{gray!6}{40} & \cellcolor{gray!6}{40} & \cellcolor{gray!6}{40} & \cellcolor{gray!6}{40} & \cellcolor{gray!6}{40} & \cellcolor{gray!6}{30} & \cellcolor{gray!6}{30} & \cellcolor{gray!6}{30} & \cellcolor{gray!6}{25} & \cellcolor{gray!6}{25} & \cellcolor{gray!6}{80} & \cellcolor{gray!6}{60} & \cellcolor{gray!6}{25} & \cellcolor{gray!6}{25} & \cellcolor{gray!6}{25} & \cellcolor{gray!6}{25} & \cellcolor{gray!6}{25} & \cellcolor{gray!6}{15} & \cellcolor{gray!6}{15}\\
BRA & 40 & 40 & 40 & 40 & 40 & 40 & 40 & 30 & 30 & 30 & 25 & 25 & 80 & 60 & 25 & 25 & 25 & 25 & 25 & 15 & 15\\
\cellcolor{gray!6}{MDE} & \cellcolor{gray!6}{40} & \cellcolor{gray!6}{40} & \cellcolor{gray!6}{40} & \cellcolor{gray!6}{40} & \cellcolor{gray!6}{40} & \cellcolor{gray!6}{40} & \cellcolor{gray!6}{40} & \cellcolor{gray!6}{30} & \cellcolor{gray!6}{30} & \cellcolor{gray!6}{30} & \cellcolor{gray!6}{25} & \cellcolor{gray!6}{25} & \cellcolor{gray!6}{80} & \cellcolor{gray!6}{60} & \cellcolor{gray!6}{25} & \cellcolor{gray!6}{25} & \cellcolor{gray!6}{25} & \cellcolor{gray!6}{25} & \cellcolor{gray!6}{25} & \cellcolor{gray!6}{15} & \cellcolor{gray!6}{15}\\
AFR & 40 & 40 & 40 & 40 & 40 & 40 & 40 & 30 & 30 & 30 & 25 & 25 & 80 & 60 & 25 & 25 & 25 & 25 & 25 & 15 & 15\\
\addlinespace
\cellcolor{gray!6}{RAS} & \cellcolor{gray!6}{40} & \cellcolor{gray!6}{40} & \cellcolor{gray!6}{40} & \cellcolor{gray!6}{40} & \cellcolor{gray!6}{40} & \cellcolor{gray!6}{40} & \cellcolor{gray!6}{40} & \cellcolor{gray!6}{30} & \cellcolor{gray!6}{30} & \cellcolor{gray!6}{30} & \cellcolor{gray!6}{25} & \cellcolor{gray!6}{25} & \cellcolor{gray!6}{80} & \cellcolor{gray!6}{60} & \cellcolor{gray!6}{25} & \cellcolor{gray!6}{25} & \cellcolor{gray!6}{25} & \cellcolor{gray!6}{25} & \cellcolor{gray!6}{25} & \cellcolor{gray!6}{15} & \cellcolor{gray!6}{15}\\
RAL & 40 & 40 & 40 & 40 & 40 & 40 & 40 & 30 & 30 & 30 & 25 & 25 & 80 & 60 & 25 & 25 & 25 & 25 & 25 & 15 & 15\\
\bottomrule
\insertTableNotes
\end{longtable}
\end{ThreePartTable}
\endgroup{}


% LR_export : learning rates
\begingroup\fontsize{8}{10}\selectfont

\begin{ThreePartTable}
\begin{TableNotes}[para]
\item \underline{\textit{Sources:}} 
\item IEA (2021) Power generation assumptions in the Stated Policies and SDS Scenarios in the World Energy Outlook 2021 - Assumptions for additional technologies and regions in the Stated Policies and Net Zero scenarios
\end{TableNotes}
\begin{longtable}[t]{rrrrrrrrrrrrrrrrrrrrr}
\caption{Learning rates, in \%}\\
\toprule
PFC & PSS & ICG & CGS & SUB & USC & UCS & GGT & GGS & GGC & OCT & OGC & HYD & NUC & CSP & WND & WNO & CPV & RPV & BIGCC & BIGCCS\\
\midrule
\cellcolor{gray!6}{0} & \cellcolor{gray!6}{0} & \cellcolor{gray!6}{0.1} & \cellcolor{gray!6}{0} & \cellcolor{gray!6}{0} & \cellcolor{gray!6}{0} & \cellcolor{gray!6}{0.05} & \cellcolor{gray!6}{0} & \cellcolor{gray!6}{0} & \cellcolor{gray!6}{0.08} & \cellcolor{gray!6}{0} & \cellcolor{gray!6}{0} & \cellcolor{gray!6}{0} & \cellcolor{gray!6}{0} & \cellcolor{gray!6}{0.1} & \cellcolor{gray!6}{0.05} & \cellcolor{gray!6}{0.15} & \cellcolor{gray!6}{0.2} & \cellcolor{gray!6}{0.2} & \cellcolor{gray!6}{0} & \cellcolor{gray!6}{0}\\
\bottomrule
\insertTableNotes
\end{longtable}
\end{ThreePartTable}
\endgroup{}


\subsection{Sensibility analysis : logit exponent}

Figure X shows the electricity market share for variable renewable energy sources (solar and wind, including CSP) under different values for the $\gamma$ parameter corresponding to the exponent of the first logit nest (\ref{fig:suminv}). The value used by default in IMACLIM-R is $\gamma = 3$. 

\begin{figure}[H]
    \centerline{\includegraphics[scale=0.5]{figures&tables/logit_MSH.png}}
    \caption{Regional disaggregation of IMACLIM-R model,  } 
    \label{fig:sensiblog}
\end{figure}

\end{landscape}


\subsection{Regional disaggregation}

\begin{figure}[H]
    \centerline{\includegraphics[scale=0.6]{figures&tables/Reg_IMC.png}}
    \caption{Regional disaggregation of IMACLIM-R model, \cite{Bibas2016} } 
    \label{fig:disag}
\end{figure}
